\documentclass{beamer}
\usepackage{beamerthemeLuebeck}

\usepackage{url}
\usepackage{pgf}

\include{tango}

\lstloadlanguages{[ANSI]C,Java,ML,Prolog,Python,Ruby,SQL,XML}

\lstdefinelanguage{python_new}%
  {morekeywords={access,and,assert,break,class,continue,def,del,elif,else,%
      except,exec,finally,for,from,global,if,import,in,is,lambda,not,%
      or,pass,print,raise,return,try,while,True,False,yield},%
   sensitive=true,%
   showspaces=false,%
   showstringspaces=false,%
   showtabs=false,%
   morecomment=[l]\#,%
   morecomment=[s]{'''}{'''},% used for documentation text
   morecomment=[s]{"""}{"""},% added by Philipp Matthias Hahn
   morestring=[b]',%
   morestring=[b]"%
  }

\newcommand{\pythoncsp}{\texttt{python-csp}}
\usepackage{zed-csp}
\usepackage{verbatim}

\title{\pythoncsp{}: bringing OCCAM to Python}
\author{Sarah Mount}
\date{Europython 2010}

%%% MEDIA
%%% GRAPHICS

\pgfdeclareimage[interpolate=true,width=8cm]{csp-logo}{./figures/pythoncsp-logo}
\pgfdeclareimage[interpolate=true,width=5cm]{hoare}{./figures/thoare}
\pgfdeclareimage[interpolate=true,width=7cm]{inmos}{./figures/transputer}

\pgfdeclareimage[interpolate=true,width=7cm]{causality}{./figures/causality}
\pgfdeclareimage[interpolate=true,width=8cm]{cramming}{./figures/cramming}
\pgfdeclareimage[interpolate=true,width=8cm]{oscope}{./figures/Oscilloscope}
\pgfdeclareimage[interpolate=true,width=8cm]{tm}{./figures/tm}
\pgfdeclareimage[interpolate=true,width=8cm]{wiring}{./figures/wiring}
\pgfdeclareimage[interpolate=true,width=8cm]{mandel}{./figures/mandelbrot}
\pgfdeclareimage[interpolate=true,width=10cm]{procs}{./figures/oscope-procs}


\usepackage{multimedia}

%%% START HERE ...
\begin{document}

\frame{\titlepage}

\frame{\tableofcontents}


\frame
{
  \pgfuseimage{csp-logo}
  \begin{block}{This talk\ldots{}}
    is all about the \pythoncsp{} project. There is a
    similar project called PyCSP, these will merge over the next few
    months. Current code base is here:
    \url{http://code.google.com/p/python-csp} \\Please do contribute bug
    fixes / issues / code / docs / rants / \ldots{}
  \end{block}
}


\section{Ancient history}

\frame
{
\pgfuseimage{inmos}
\begin{block}{INMOS B0042 Board \copyright{} David May}
The Transputer was the original ``multicore'' machine and was built to
run the OCCAM language -- a realisation of CSP.
\url{http://www.cs.bris.ac.uk/~dave/transputer.html}\\
\end{block}
}


\frame
{
  \pgfuseimage{hoare}
  \begin{block}{Tony Hoare}
    Invented CSP, OCCAM, Quicksort and other baddass stuff. You may
    remember him from such talks as his Europython 2009 Keynote.
  \end{block}
}



\frame { 

  \frametitle{OCCAM} 

  OCCAM lives on in it's current implementation as OCCAM-$\pi$
  \url{http://www.cs.kent.ac.uk/projects/ofa/kroc/}


  It can run on Lego bricks, Arduino boards and other things. Like
  Python it uses indentation to demark scope. Unlike Python OCCAM is
  MOSTLY IN UPPERCASE.
}


\frame
{
  \frametitle{CSP and message passing}
  \framesubtitle{This next bit will be revision for many of you!}
  \pgfuseimage{cramming}
}


\frame
{
  \frametitle{Most of us think of programs like this}
  \pgfuseimage{tm}
  \begin{block}{}
  \url{http://xkcd.com/205/}    
  \end{block}
}

\frame
{
  \frametitle{This is the old ``standard'' model!}

  \begin{itemize}
  \item Multicore will soon take over the world (if it hasn't already).
  \item Shared memory concurrency / parallelism is hard:
    \begin{itemize}
    \item Race hazards
    \item Deadlock
    \item Livelock
    \item Tracking which lock goes with which data and what order
      locks should be used.
    \item Operating Systems are old news!
    \end{itemize}
  \end{itemize}
}

\frame
{
  \frametitle{Message passing}

  \begin{definition}
    In \emph{message passing} concurrency (or parallelism)
    ``everything'' is a process and no data is shared between
    processes. Instead, data is ``passed'' between \emph{channels}
    which connect processes together. Think: OS processes and UNIX
    pipes.
  \end{definition}

  \begin{itemize}
  \item The \emph{actor} model (similar to Kamaelia) uses asynchronous
    channels. Processes write data to a channel and continue execution. 
  \item CSP (and $\pi$-calculus) use synchronous channels, or
    \emph{rendezvous}, where a process writing to a channel has to
    wait until the value has been read by another process before
    proceeding.
  \end{itemize}
}

\frame
{
  \frametitle{Process diagram}
  \pgfuseimage{procs}
}


\frame
{
  \frametitle{Why synchronous channels?}
  \pgfuseimage{causality}
  \begin{block}{}
    Leslie Lamport (1978). Time, clocks, and the ordering of events in a
    distributed system". Communications of the ACM 21 (7)
  \end{block}
}

%%%%%%%%%%%%%%%%%%%%%%%%%%%%%%%%%%%%%%%%%%%%%%%%%%%%%%%%%%%%%%%%%%%%%%%%%%%%
\section{\pythoncsp{}: features and idioms}

\subsection{Process creation}

\begin{frame}[fragile]
  \frametitle{Hello World!}
  \begin{lstlisting}[language=python_new]
from csp.cspprocess import * 

@process
def hello():
    print 'Hello CSP world!'

hello().start()
  \end{lstlisting}
\end{frame}



\begin{frame}[fragile]
  \frametitle{Example of process creation: web server}
  \begin{lstlisting}[language=python_new]
@process
def server(host, port):
  sock = socket.socket(...)
  # Set up sockets
  while True:
      conn, addr = sock.accept()
      request = conn.recv(4096).strip()
      if request.startswith('GET'):
          ok(request,conn).start()
      else:
          not_found(request,conn).start()
  \end{lstlisting}
\end{frame}


\begin{frame}[fragile]
  \frametitle{Example of process creation: web server}
  \begin{lstlisting}[language=python_new]
@process
def not_found(request, conn):
    page = '<h1>File not found</h1>'
    conn.send(response(404, 
                       'Not Found', 
                       page))
    conn.shutdown(socket.SHUT_RDWR)
    conn.close()
    return
  \end{lstlisting}
\end{frame}


\subsection{Parallel, sequential and repeated processes}

\begin{frame}[fragile]
  \frametitle{Combining processes: in parallel}
  \begin{lstlisting}[language=python_new]
    # With a Par object:
    Par(p1, p2, p3, ... pn).start()
    # With syntactic sugar:
    p1 // (p2, p3, ... pn)
    # RHS must be a sequence type.
  \end{lstlisting}
\end{frame}


\begin{frame}[fragile]
  \frametitle{Combining processes: repeating a process sequentially}
  \begin{lstlisting}[language=python_new]
@process
def hello():
    print 'Hello CSP world!'
if __name__ == '__main__':
    hello() * 3
    2 * hello()
  \end{lstlisting}
\end{frame}


\subsection{Communication via channel read / writes}

\begin{frame}[fragile]
  \frametitle{Channels}
  \begin{lstlisting}[language=python_new]
@process
def client(chan):
    print chan.read()
@process
def server(chan):
    chan.write('Hello CSP world!')

if __name__ == '__main__':
    ch = Channel()
    Par(client(ch), server(ch)).start()
  \end{lstlisting}
\end{frame}


\frame
{
  \frametitle{What you need to know about channels}

  \begin{itemize}
  \item Channels are currently UNIX pipes
  \item This will likely change
  \item Channel rendezvous is ``slow'' compared with JCSP:\\
    200$\mu{}s$ \emph{vs} 16$\mu{}s$
  \item This will be fixed by having channels written in C
  \item Because channels are, at the OS level, open ``files'', there
    can only be a limited number of them (around $300$ on my machine)
  \item This makes me sad :-(
  \end{itemize}
}

\subsection{More channels: ALTing and choice}

\begin{frame}[fragile]
  \frametitle{Message passing an responsiveness}

  \begin{itemize}
  \item This isn't about speed.
  \item Sometimes, you have several channels and reading from them all
    round-robin may reduce responsiveness if one channel read blocks
    for a long time.
  \end{itemize}
This is BAD:
  \begin{lstlisting}[language=python_new]
@process
def foo(channels):
    for chan in channels:
        print chan.read()
  \end{lstlisting}
\end{frame}


\begin{frame}[fragile]
  \frametitle{ALTing and choice}
  \begin{definition}
    ALTing, or non-deterministic choice, \emph{selects} a channel to
    read from from those channels which are ready to read from.
  \end{definition}
You can do this with (only) two channels:
  \begin{lstlisting}[language=python_new]
@process
def foo(c1, c2):
    print c1 | c2
    print c1 | c2
  \end{lstlisting}
\end{frame}


\begin{frame}[fragile]
  \frametitle{ALTing proper (many channels)}
  \begin{lstlisting}[language=python_new]
@process
def foo(c1, c2, c3, c4, c5):
    alt = Alt(c1, c2, c3, c4, c5)
    # Choose random available offer
    print alt.select()
    # Avoid last selected channel
    print alt.fair_select() 
    # Choose channel with lowest index
    print alt.priselect()   
  \end{lstlisting}
\end{frame}


\begin{frame}[fragile]
  \frametitle{Syntactic sugar for ALTing}
  \begin{lstlisting}[language=python_new]
@process
def foo(c1, c2, c3, c4, c5):
    gen = Alt(c1, c2, c3, c4, c5)
    print gen.next()
    print gen.next()
    print gen.next()
    print gen.next()
    print gen.next()
  \end{lstlisting}
\end{frame}


\begin{frame}[fragile]
  \frametitle{ALTing: when you really, really must return right now}
  \begin{lstlisting}[language=python_new]
@process
def foo(c1, c2, c3):
    # Skip() will /always/ complete 
    # a read immediately
    gen = Alt(c1, c2, c3, Skip())
    print gen.next()
    ...
  \end{lstlisting}
\end{frame}


\subsection{More channels: poisoning}

\begin{frame}[fragile]
  \frametitle{Channel poisoning}
  \begin{definition}
    Idiomatically in this style of programming most processes contain
    infinite loops. \emph{Poisoning} a channel asks all processes
    connected to that channel to shut down automatically. Each process
    which has a poisoned channel will automatically poison any other
    channels it is connected to. This way, a whole graph of connected
    processes can be terminated, avoiding race conditions.
  \end{definition}
  \begin{lstlisting}[language=python_new]
@process
def foo(channel):
    ...
    channel.poison()
  \end{lstlisting}
\end{frame}


\subsection{Producer / consumer or worker / farmer models}


\frame
{
  \frametitle{A bigger example: Mandelbrot generator}
  \pgfuseimage{mandel}
}


\begin{frame}[fragile]
\begin{lstlisting}[language=python_new]
@process
def main(IMSIZE):
    channels = []
    # One producer + channel for each image column.
    for x in range(IMSIZE[0]): 
        channels.append(Channel())
        processes.append(mandelbrot(x, ... channels[x]))
    processes.insert(0, consume(IMSIZE, channels))
    mandel = Par(*processes)
    mandel.start()
\end{lstlisting}
\end{frame}

\begin{frame}[fragile]
\begin{lstlisting}[language=python_new]
@process
def mandelbrot(xcoord, cout):
    # Do some maths here
    cout.write(xcoord, column_data)
\end{lstlisting}
\end{frame}


\begin{frame}[fragile]
\begin{lstlisting}[language=python_new]
@process
def consumer(cins):
    # Initialise PyGame...
    gen = len(cins) * Alt(*cins)
    for i in range(len(cins)):
        xcoord, column = gen.next()
        # Blit that column to screen
    for event in pygame.event.get():
        if event.type == pygame.QUIT:
            for channel in cins: 
                channel.poison()
            pygame.quit()
\end{lstlisting}
\end{frame}


\section{Using built-in processes}

\frame
{
  \frametitle{Oscilloscope}
  \pgfuseimage{oscope}
}

\begin{frame}[fragile]
\begin{lstlisting}[language=python_new]
@process
def Oscilloscope(inchan):
    # Initialise PyGame
    ydata = []
    while not QUIT:
        ydata.append(inchan.read())
        ydata.pop(0)
        # Blit data to screen...
    # Deal with events
\end{lstlisting}
\end{frame}

\begin{frame}[fragile]
\begin{lstlisting}[language=python_new]
from csp.builtins import Sin
def trace_sin():
    chans = Channel(), Channel()
    Par(GenerateFloats(chans[0]),
        Sin(chans[0], chans[1]), 
        Oscilloscope(chans[1])).start()
    return    
\end{lstlisting}
\end{frame}


\frame
{
  \frametitle{Process diagram}
  \pgfuseimage{procs}
}



\begin{frame}[fragile]
\begin{lstlisting}[language=python_new]
def trace_mux():
    chans = [Channel() for i in range(6)]
    par = Par(GenerateFloats(chans[0]),
              Delta2(chans[0], chans[1], 
                     chans[2]),
              Cos(chans[1], chans[3]),
              Sin(chans[2], chans[4]),
              Mux2(chans[3], chans[4], 
                   chans[5]),
              Oscilloscope(chans[5]))
    par.start()
\end{lstlisting}
\end{frame}


\section{Future challenges}

\frame
{
  \frametitle{Wiring}
  \pgfuseimage{wiring}
}


\end{document}
 
