\documentclass[compress]{beamer}
\usetheme{Boadilla}

\usepackage{
    url,
    booktabs,
    pgf,
    tabularx,
    verbatim,
}

\include{tango} % Uses listings package.

\lstloadlanguages{[ANSI]C,Java,ML,Prolog,Python,Ruby,SQL,erlang,XML}

\newcommand{\pythoncsp}{\texttt{python-csp}}
\usepackage{zed-csp}

\title{Message-passing concurrency in Python}
\author{Sarah Mount -- @snim2}
\date{Europython 2014}

\AtBeginSection{\frame{\sectionpage}}

%%% MEDIA
%%% GRAPHICS

\pgfdeclareimage[interpolate=true,width=8cm]{csp-logo}{./figures/pythoncsp-logo}
\pgfdeclareimage[interpolate=true,width=5cm]{hoare}{./figures/thoare}
\pgfdeclareimage[interpolate=true,width=7cm]{inmos}{./figures/transputer}

\pgfdeclareimage[interpolate=true,width=7cm]{causality}{./figures/causality}
\pgfdeclareimage[interpolate=true,width=8cm]{cramming}{./figures/cramming}
\pgfdeclareimage[interpolate=true,width=8cm]{oscope}{./figures/Oscilloscope}
\pgfdeclareimage[interpolate=true,width=8cm]{tm}{./figures/tm}
\pgfdeclareimage[interpolate=true,width=8cm]{wiring}{./figures/wiring}
\pgfdeclareimage[interpolate=true,width=8cm]{mandel}{./figures/mandelbrot}
\pgfdeclareimage[interpolate=true,width=10cm]{procs}{./figures/oscope-procs}


\usepackage{multimedia}

%%% START HERE ...
\begin{document}

\logo{\pgfuseimage{uni}}
\frame{\titlepage}
\frame{\tableofcontents}

\section{Why multicore?}

\frame{    
    \frametitle{Why multicore? (1/2)}
    Feature size (mm) vs time (year). CPU sizes are shrinking!
    Get the raw data here: \url{http://cpudb.stanford.edu/}
    \centering{\pgfuseimage{moores-law}}
}

\frame{    
    \frametitle{Why multicore? (2/2)}
    New platforms, such as this board from Adapteva Inc. which reportedly
    runs at 50GFPLOPS/Watt. Photo \copyright Adapteva Inc.
    \centering{\pgfuseimage{parallella}}
}


\section{Message passing: an idea whose time has come (back)}

\frame{    
    \frametitle{Message passing concurrency}
    \pgfuseimage{hoare}
    \begin{block}{Tony Hoare}
    CSP\footnote{other process algebras are available} (Communicating 
    Sequential Processes) was invented by Tony Hoare in the late 70s / early 
    80s as a response to the difficulty of reasoning about concurrent and 
    parallel code. It was implemented as the OCCAM programming language
    for the INMOS Transputer.
    \end{block}
}

\frame{    
    \frametitle{Message passing concurrency}
    CSP\footnote{other process algebras are available} take the view 
    that computation consists of \alert{communicating} processes which are 
    individually made up of \alert{sequential} instructions. The communicating 
    processes all run "at the same time" and \alert{share no data}. Because
    they do not share data they have to cooperate by \alert{synchronising} on 
    \alert{events}. A common form of synchronisation (but not the only one)
    is to pass data between processes via \alert{channels}.\\
    ~\\
    \textbf{THINK} UNIX processes communicating via pipes.\\
    ~\\
    \textbf{BEWARE} CSP is an abstract ideas which use overloaded jargon, 
    a \emph{process} or an \emph{event} need not be reified as an OS process or
    GUI event\ldots
}


\frame{    
    \frametitle{Benefits of message passing}
    Threads and locking are tough! Deadlocks, starvation, race hazards are tough.
    Locks do not compose easily, errors in locking cannot easily be 
    recovered from, finding the right level of granularity is hard.\\
    ~\\
    CSP does not exhibit race hazards, there are no locks, the hairy 
    details are all hidden away in either a language runtime or a library. WIN! 
}


\section{Message passing: all the cool kids are doing it}

\begin{frame}[fragile]
    \frametitle{UNIX}
    \begin{lstlisting}[language=sh]
$ cat mydata.csv | sort | uniq | wc -l
    \end{lstlisting}
\end{frame}

\begin{frame}[fragile]
    \frametitle{Go}
    \begin{lstlisting}[language=C,morekeywords={func,make,go}]
func main() {
     mychannel := make(chan string)
     go func() {
          mychannel <- "Hello world!"
     }()
     fmt.Println(<-mychannel)
}
    \end{lstlisting}
\end{frame}

\begin{frame}[fragile]
    \frametitle{Rust}
    \begin{lstlisting}[language=Java,morekeywords={fn,let,spawn}]
fn main() {
    let (chan, port) = channel();

    spawn(proc() {
        chan.send("Hello world!");
    });

    println!("{:s}", port.recv());
}
    \end{lstlisting}
\end{frame}

\begin{frame}[fragile]
    \frametitle{Scala}
    \begin{lstlisting}[language=Java,morekeywords={val,def,receive}]
object Example {
  class Example extends Actor {
    def act = {
      receive {
        case msg => println(msg)
      }
    }
  }
  def main(args:Array[String]) {
    val a = new Example 
    a.start
    a ! "Hello world!"
  }
}
    \end{lstlisting}
\end{frame}

\begin{frame}[fragile]
    \frametitle{python-csp\footnote{http://github.com/snim2/python-csp}}
    \begin{lstlisting}[language=python]
@process
def client(chan):
    print chan.read()
@process
def server(chan):
    chan.write('Hello world!')
    
if __name__ == '__main__':
    ch = Channel()
    Par(client(ch), server(ch)).start()
    \end{lstlisting}
\end{frame}

\begin{frame}[fragile]
    \frametitle{naulang\footnote{Samuel Giles (2014) Is it possible to build a performant, dynamic language that supports concurrent models of computation? University of Wolverhampton BSc thesis}: building on the RPython toolchain}
    \begin{lstlisting}[language=python,morekeywords={let,fn,async,print,chan}]
let f = fn(mychannel) {
    mychannel <- "Hello world!"
}
let mychannel = chan()
async f(mychannel)
print <: mychannel
    \end{lstlisting}
\end{frame}

\frame{
    \frametitle{naulang\footnote{Samuel Giles (2014) Is it possible to build a performant, dynamic language that supports concurrent models of computation? University of Wolverhampton BSc thesis}: building on the RPython toolchain}
\begin{table}[]
	\caption{Results from a tokenring benchmark @SamuelGiles\_}
	\begin{tabular}[]{l|rrrr}
    \toprule
    ~ & \textbf{Min ($\mu$s)} & \textbf{Max ($\mu$s)} & \textbf{Mean ($\mu$s)} & \textbf{s.d.}\\
    \midrule
    Go 1.1.2 & 99167 & 132488 & 100656 & 1590\\
    \midrule
    Occam-$\pi$ & 68847 & 73355 & 69723 & 603\\
    \midrule
    naulang & 443316 & 486716 & 447894 & 6507\\
    naulang JIT & 317510 & 589618 & 322304 & 10613\\
    naulang (async) & 351677 & 490241 & 354325 & 6401\\
    naulang JIT (async) & 42993 & 49496 & 44119 & 295\\
    \bottomrule	
    \end{tabular}
    \centering{\url{http://github.com/samgiles/naulang}}
\end{table}
}

\frame{
    \frametitle{Aside: Did those benchmarks matter?}
    That was a comparison between naulang and two compiled languages. The 
    benchmark gives us some reason to suppose that a dynamic language
    (built on the RPython chain) can perform reasonably well compared to
    static languages.\\
    ~\\
    Does this matter? We don't want message passing to become a bottleneck
    in optimising code\ldots
}

\section{Design choices in message passing languages and libraries}

\frame{    
    \frametitle{Synchronous vs. asynchronous channels}
    
    \textbf{Synchronous} channels block on read / write - examples include 
    OCCAM, JCSP, python-csp, UNIX pipes. \textbf{Asynchronous} channels are 
    non-blocking - examples include Scala, Rust. Some languages give you both
    - examples include Go.\\
    
    \textbf{Synchronous} channels are easier to reason about (formally) and
    many people believe are easier to use. \\
    ~\\
    \textbf{Asynchronous} channels are sometimes thought to be faster. Like
    most claims about speed, this isn't always true.
}

\begin{frame}[fragile]    
    \frametitle{Uni or bidirectional channels? Point to point?}

    The JCSP\footnote{\url{http://www.cs.kent.ac.uk/projects/ofa/jcsp/}} project
    (for Java) separates out different sorts of channels types, e.g.:
    \begin{itemize}
    \item \verb!AnyToAnyChannel!
    \item \verb!AnyToOneChannel!
    \item \verb!OneToAnyChannel!
    \item \verb!OneToOneChannel!
    \end{itemize}
    This isn't very Pythonic, but it can catch errors in setting up a network
    of channels in a message passing system.
\end{frame}

\frame{    
    \frametitle{Mobile or immobile channels}
    
    A \textbf{mobile} channels can be sent over another channel. This means 
    that the network of channels can change its topology dynamically at runtime.
    Why might we want to do this? 
    \begin{itemize}
    \item our process are actually running over a network of machines or on
    a manycore processor. We want to migrate some processes to perform load 
    balancing or due to some failure condition within a network.
    \item some processes have finished their work and any channels associated
    with them need to be either \alert{poisoned} (destroyed) or re-routed 
    elsewhere.
    \end{itemize}
}

\frame{    
    \frametitle{Reifying sequential processes}
    In \textbf{CSP} and \textbf{actor model} languages the paradigm is of 
    a number of \emph{sequential processes} which run in parallel and 
    communicate via channels (think: UNIX processes pipes). It is up to the 
    language designer how to reify those sequential processes. Common choices are:
    \begin{description}
    \item [1 CSP process is 1 coroutine] \alert{very fast message passing} (OCCAM-$\pi$ 
    reports a small number of ns), cannot take advantage of multicore.
    \item [1 CSP process is 1 OS thread] slower message passing; \alert{can take advantage of multicore}
    \item [1 CSP process is 1 OS process] very slow message passing; \alert{can migrate CSP processes across a network to other machines}
    \item [1 CSP processes is 1 coroutine] many coroutines are multiplexed onto an OS thread
    \item [1 CSP processes is 1 OS thread] many OS threads are multiplexed onto an OS process
    \item [etc.]
    \end{description} 
}

\section{Message passing in Python too!}

\frame{    
    \frametitle{So many EP14 talks!}
    Several EP14 talks involve coroutines (lightweight threads) and / or channels:
    \begin{description}
    \item [Daniel Pope] Wedneday 11:00
    \item [Christian Tismer, Anselm Kruis] Wednesday 11:45
    \item [Richard Wall] Friday 11:00
    \item [Benoit Chesneau] Friday 12:00
    \item [Dougal Matthew] Friday 12:00
    \end{description} 
    Message-passing has already entered the Python ecosystem and there are
    constructs such as pipes and queues in the \texttt{multiprocessing} library.
    Could we do better by adopting some ideas from other languages?
}


\frame{    
    \frametitle{The (near) future}
    Python for the Parallella will be coming out this Summer, thanks to funding
    from the RAE. If you are interested in this, please keep an eye on the
    \alert{futurecore}\footnote{http://github.com/futurecore/} GitHub
    organization.\\
    ~\\
    python-csp is moving back into regular development. Will it ever be 
    fast?
}


\frame{
  \frametitle{Thank you.}
}

\end{document}
